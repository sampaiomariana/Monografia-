Processo de definição de arquitetura de software consiste de processo relevante em desenvolvimento de software. A arquitetura de software é um processo importante no momento do desenvolvimento do software. Esse processo pode englobar atividades com os seguintes propósitos: definição, avaliação, descrição, manutenção, certificação e implementação de arquitetura de software. Ao desenvolver a arquitetura do software é possível determinar as atividades de definir, documentar, manter, melhorar e certificar a implementação da arquitetura. Esse processo permite definir esbelecida a arquitetura do software. Nesse contexto, arquitetura de software é estrutura de software. Estrutura essa composta por elementos que integram o software estrutura do sistema de software. Estrutura essa que irá relacionar todos os componentes que fazem parte do sistema de software fazendo com que essa estrutura arquitetura seja a base do sistema 
~\cite{ISO_1471}.

Considerando a relevância de processo de definição de arquitetura de software em desenvolvimento de software, este trabalho enfoca a configuração de processo de desenvolvimento de software com elementos típicos de processo de definição de arquitetura de software. Particularmente, atividades relacionadas à descrição e à avaliação de arquitetura de software. Ao longo deste trabalho, diversas fontes de informação serão acessadas, particularmente normas \emph{(standards)} acerca de descrição de arquitetura de software e descrições de métodos para avaliação de arquitetura de software.


Devido à importância desse processo, o foco deste trabalho será em apresentar as vantagens de desenvolver a arquitetura de software seguindo um processo de configuração da arquitetura de software. Para isso, serão reunidas configurações que são abordadas pelo \emph{\acrfull{OpenUp}} e pela \emph{ISO 1471-2000 - IEEE Recommended Practice for Architectural Description of Software-Intensive Systems} a fim de apresentar uma documentação e uma implementação mais completa da arquitetura de software~\cite{ISO_1471}~\cite{openup}.Com isso, a partir da configuração desenvolvida é possível definir a arquitetura de software e avaliá-la através do método \emph{\acrfull{SAAM}}.Esse método apresenta ao arquiteto os pontos problemáticos da arquitetura dessa forma é possível propor novas soluções para ajustar a arquitetura ao que é necessário para que o sistema de software tenha o seu pleno funcionamento~\cite{survey_methods}~\cite{scenario_methods}.

Devido à complexidade que está envolvida para determinar a arquitetura adequada o processo apresentado configurado será seguido em todos os pontos. Sendo assim, será determinado se a arquitetura que foi escolhida está de acordo com o que o sistema de software necessita. Essa avaliação será feita através do método \emph{\acrfull{SAAM}} que envolve analisar os cenários e classificá-los, a partir desses cenários deve ser realizada a avaliação tanto individual quanto interativa do cenário~\cite{survey_methods}~\cite{scenario_methods}.

Com isso, os pontos fracos e pontos fortes da arquitetura são levantados e analisados ao projetar a arquitetura seguindo um processo, pois dessa forma os pontos problemáticos podem ser vistos antes, podendo assim melhorar a implementação da arquitetura no momento em que ela é levada para a prática~\cite{survey_methods}~\cite{scenario_methods}.


\section{Objetivo geral}

Configurar processo de desenvolvimento de software com elementos de processo de definição de arquitetura de software e exemplificar uso do processo configurado.

\section{Objetivos específicos}

Os objetivos específicos do projeto são:
\begin{itemize}
    \item Descrever  conceitos acerca de arquitetura de software;
    \item Descrever elementos do processo de definição de arquitetura de software;
    \item Descrever métodos para avaliação de arquitetura de software;
    \item Configurar processo de desenvolvimento de software com elementos de processo de definição de arquitetura de software;
    \item Exemplificar o uso de processo de desenvolvimento de software configurado.
\end{itemize}

\section{Motivações e justificativas}

Em desenvolvimento de software, existem diversas atividades e artefatos relacionados à arquitetura de software. Em desenvolvimento de software, é relevante promover o acesso à informação acerca dessas atividades e artefatos. Pode-se promover o acesso a essa informação disponibilizando-a na descrição de processo de desenvolvimento de software.

\section{Estrutura do documento}

Esta monografia está dividida em fundamentação teórica, estudo de caso para um exemplo de configuração de uso e conclusão.A fundamentação teórica é composta por quatro capítulos, para o desenvolvimento da fundamentação teórica foram realizadas diversas consultas à diversas fontes de informação.

No capítulo dois são apresentados os conceitos que são base para o entendimento deste trabalho, dentre esses conceitos tem-se a definição de arquitetura de software, descrição da arquitetura de software, usos para a descrição da arquitetura de software, ponto de vista e visão de arquitetura de software, práticas para a descrição de arquitetura de software e arcabouços de arquitetura.No capítulo três são realizadas as descrições sobre o processo de definição de arquitetura de software. Nesse capítulo é realizada a definição do ciclo de vida de software, processos em ciclo de vida de software, processos técnicos em ciclo de vida de software e processo de definição de arquitetura de software.
No capítulo quatro são realizadas as descrições sobre os métodos para avaliação de arquitetura de software. Nesse capítulo é realizada a definição para a avaliação de arquitetura de software, métodos para a avaliação de arquitetura de software, descrição sobre o \emph{\acrfull{SACAM}}, \emph{\acrfull{ATAM}} e o \emph{\acrfull{SAAM}}. 

A descrição  para o exemplo de configuração de uso tem sua descrição realizada em dois capítulos, sendo que no capítulo cinco nele que é realizado o exemplo de configuração de processo de desenvolvimento e no capítulo seis é descrito o exemplo de uso de processo de desenvolvimento.

No capítulo sete encontram-se as avaliações em relação ao projeto prático diante da configuração e uso de processo de desenvolvimento do software. Por fim, no capítulo oito é apresentada a conclusão, nela são levantadas as considerações acerca do trabalho desenvolvido e são sugeridos futuros trabalhos possíveis.