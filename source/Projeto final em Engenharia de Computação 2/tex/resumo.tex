No desenvolvimento de software existem processos que podem ser adotados na definição de arquitetura de software. Este trabalho visa promover o entendimento de processo de definição de arquitetura de software por meio da configuração de processo de desenvolvimento de software com elementos de processo de definição de arquitetura de software e uso do processo de desenvolvimento de software configurado. Tendo em vista ter uma configuração de processo com maior detalhamento das informações em relação ao sistema de software desenvolvido, neste trabalho foi realizada uma configuração de processo  reunindo dois elementos já existentes. Sendo estes, o processo  \emph{\acrfull{OpenUP}} e a descrição da arquitetura proposta pelo \emph{IEEE 1471-2000 - IEEE Recommended Practice for Architectural Description of Software-Intensive Systems}. A partir da união desses dois elementos foi construído um primeiro artefato chamado de  prática para descrição de arquitetura. Tendo a descrição da arquitetura documentada, tem-se a segunda etapa da proposta de configuração deste trabalho que é realizar a avaliação da arquitetura candidata a fim de verificar se ela está adequada para a implementação do sistema de software usando o \emph{\acrfull{SAAM}}, este método visa analisar a arquitetura de software a partir de cenários. Podendo assim encontrar problemas prévios na arquitetura antes da implementação do sistema. Com isso, foi possível observar que a identificação desses problemas permite obter uma implementação com menor retrabalho, já que erros terão sido investigados e analisados anteriormente, os componentes da arquitetura estarão detalhados de melhor forma além de permitir que se tenha a documentação de atividades que foram realizadas durante o desenvolvimento do sistema de software.