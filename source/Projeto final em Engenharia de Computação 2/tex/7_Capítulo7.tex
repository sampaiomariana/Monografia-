Este capítulo tem como intuito apresentar, tendo como base o referencial teórico e o exemplo desenvolvido, a conclusão do trabalho. Para isso, ele foi dividido em duas partes. A primeira apresenta considerações finais e a segunda apresenta sugestões para trabalhos futuros relacionados ao tema abordado.

\section{Considerações Finais}

O principal objetivo deste trabalho foi configurar processo de desenvolvimento de software utilizando elementos de processo de definição da arquitetura de software e exemplificar o uso do processo de desenvolvimento configurado. O processo configurado foi o OpenUP.  A configuração do processo de desenvolvimento consistiu em incluir prática para descrever arquitetura de software recomendada em \emph{ISO 1471-2000 - IEEE Recommended Practice for Architectural Description of Software-Intensive Systems}, e incluir, em tarefas do OpenUP, passos (steps) para analisar arquitetura de software por meio do método de análise de arquitetura \emph{\acrfull{SAAM}}.

Analisando a configuração do processo a fim de obter mais informações e detalhamento do que o sistema de software deveria prover para o usuário além de realizar a avaliação da arquitetura antes da implementação, pode-se observar que foi de grande valia ter utilizado um processo configurado para desenvolver o sistema de software em questão. Dentre as vantagens que podem ser abordadas tem-se a interação que é promovida entre os desenvolvedores e as partes interessadas no sistema, tem-se o levantamento de requisitos funcionais, a descrição da arquitetura, detalhamento dos componentes da arquitetura sob uma visão \emph{(view)} e representando usando um ponto de vista \emph{(viewpoint)}, além de ter a documentação das atividades que foram realizadas durante a configuração do processo de desenvolvimento. Com isso, pode ser concluído para este trabalho que realizar a especificação da arquitetura de software seguindo um processo é de grande relevância considerando o tempo de trabalho, visto que os cenários podem ser explorados e analisados antes da sua implementação.

\section{Trabalhos Futuros}

Como sugestão de trabalho futuro relacionado a este projeto, tem-se  implementar o sistema de software, seguindo o processo de desenvolvimento proposto, com mais desenvolvedores e implementar mais funcionalidades para o sistema de software. O trabalho futuro pode também considerar ampliar o escopo. Por exemplo, implementar mais postos de saúde e maior população do banco de dados. Também pode ser ampliado o uso do sistema de software para mais usuários ao mesmo tempo. Uma outra sugestão para prosseguir este trabalho é melhorar o requisito segurança, visto que o banco de dados utilizado foi um banco de dados sem ênfase na segurança da informação. Com o aumento do escopo, melhorar a segurança seria relevante, visando assegurar a segurança do sistema frente a ataques maliciosos ou a acidentes. 


