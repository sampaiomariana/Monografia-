In software development there are processes that can be adopted in defining the software architecture. This work aims to promote understanding of the software architecture definition process, configuring the software development process with elements of the software architecture definition process and using the configured software development process. In order to have a process configuration with more detailed information regarding the developed software system, in this work a process configuration was carried out, bringing together two existing processes. 
These being the \emph{\acrfull{OpenUP}} process and the description of the architecture proposed by \emph{IEEE 1471-2000 - IEEE Recommended Practice for Architectural Description of Software-Intensive Systems}. 
A first artifact called practice for architecture description was built. With the description of the documented architecture, the second stage of the proposed configuration of this work follows, which consists of evaluating the candidate in order to verify if he is suitable for the implementation of the software system through 
\emph{\acrfull{SAAM}}, this method aims to analyze the software architecture from scenarios, that is, according to the functionality that the system has. Thus being able to find previous problems in the architecture before the implementation of the system. With this, it was possible to observe that the identification of these problems allows to obtain an implementation with less rework, since the errors will have already been investigated and analyzed previously, the components of the architecture will be detailed in a better way, besides allowing to have the documentation of all the activities that were performed during the development of the software system.